\chapter*{Resumen}


Se plantea un proyecto de Trabajo de Fin de Grado basado en el desarrollo de un videojuego de libre elección, siguiendo una pautas técnicas proporcionadas por el tutor.
Entre estas directrices se encuentran indicaciones como las siguientes:

- Deberá ser un desarrollo en dos dimensiones: Los desarrollos en 2D se recomiendan a desarrolladores principiantes ya que en primera instancia es considerablemente más fácil de desarrollar.
%\href{https://www.finalparsec.com/blog_posts/2d-vs-3d#:~:text=It\%20makes\%20more\%20sense\%20for,the\%20more\%20sophisticated\%203D\%20games.}

- Se usará un motor e desarrollo conocido por la industria del videojuego: Facilita el desarrollo proporcionando herramientas y sistemas genéricos ya creados y listos para usar sobre los que se pueden desarrollar más sistemas personalizados para nuestro juego.

- El videojuego desarrollado deberá ser por turnos: De nuevo se busca el método más sencillo para presentar un videojuego ya que eliminan componentes de sincronización de sistemas que puedan requerir de arquitecturas y sistemas más complejos que puedan repercutir negativamente en el tiempo de desarrollo, reduciendo la cantidad de funcionalidades o sistemas de juegos desarrollados en la entrega.

- El alcance del proyecto deberá ajustarse a unas 300 horas de trabajo: Es el tiempo estimado que deberían durar los Trabajos de Fin de Grado. La entrega deberá contener el mayor número de funcionalidades que se hayan podido desarrollar, así como la elaboración de una Memoria sobre el desarrollo del proyecto.

Todo esto se ha condensado en la elaboración de un videojuego de combate por turnos con cartas, desarrollado en \bibitem{Unity}.
 El juego consiste en peleas contra enemigos que controla el propio ordenador donde el jugador podrá manejar las acciones en forma de cartas que realiza un personaje elegido al inicio de la partida. Las cartas conformarán distintos "mazos" entre los que el jugador podrá elegir antes de empezar la partida.