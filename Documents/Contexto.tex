%!TEX root =  tfg.tex
\chapter{Contexto}

\begin{quotation}[Novelist]{Ernest Hemingway (1899--1961)}
The good parts of a book may be only something a writer is lucky enough to overhear or it may be the wreck of his whole damn life -- and one is as good as the other.
\end{quotation}

\begin{abstract}
Resumen de lo que va a ocurrir en el capítulo. ¿Cuál es el objetivo que tenemos con este capítulo?
\end{abstract}

\section{El mundo del Videojuego}
El videojuego es a día de hoy una de las herramientas más populares de entretenimiento o arte (como algunas personas defienden) que existe en el mundo. Su uso se expande tanto mundialmente como generacionalmente siendo los jóvenes los usuarios más comunes, aunque existen gran cantidad de videojuegos dirigidos a un público más adulto, cuestionando moralmente al jugador, proponiendo puzles complejos, definir estrategias, historias profundas y emocionantes, sobre la amistad, la pérdida. Mientras otros buscan una diversión fútil, para "pasar el rato" mientras esperas al autobús, o no apetece hacer nada.

\section{Historia de los videojuegos}
%\cite[Videojuego]{wiki:Videojuego}
 El Videojuego es un ámbito relativamente joven si se compara con otros tipos de entretenimiento que han existido a lo largo de la historia contando con apenas 75 años desde la creación de las primeras máquinas que ofrecían una experiencia interactiva sobre una pantalla que apenas podían considerarse como videojuegos.

 A lo largo de los años y sobretodo a medida que la tecnología avanzaba, los videojuegos se han vuelto uno de los métodos más comunes de entretenimiento primeramente gracias a las consolas, posteriormente ordenadores, llegando hasta hoy día que tenemos acceso a infinidad de los mismos desde nuestros dispositivos móviles.

 Junto al desarrollo de la tecnología los videojuegos también se han desarrollado considerablemente. Desde un simple partido de tenis con dos barras a los lados que mueves verticalmente y una bola que poder golpear como el Pong hasta la simulación del comportamiento de planetas y mundos imaginarios, pasando por la generación de contenido procedural (considerado como infinito muchas veces) y poder disfrutarlo en línea junto a otros jugadores.



\section{Desarrollo de Videojuegos}

El desarrollo de videojuegos se ha ido popularizando y modernizando también con los años. Los desarrolladores han ido formando empresas o grupos llamados "Estudios de videojuegos" que les daba un nombre bajo los que publicar sus creaciones. Produciendo tanto los videojuegos finales como herramientas de desarrollo de videojuegos.

A lo largo de lo años se han ido desarrollando tecnologías y el desarrollo de motores de videojuegos como Unity, Unreal Engine, CryEngine e inlcuso soluciones open source como Godot o LÖVE entre otros. Esto ha facilitado una entrada más sencilla a muchos desarrolladores que quisieran explorar este mundo del videojuego.

Finalmente, mencionar que el desarrollo de videojuegos ha podido mutar hasta lo que conocemos hoy en día como juegos independientes o "Indie" que son producciones hechas por desarrolladoras pequeñas o incluso un solo individuo. Promoviendo así esta modalidad de la informática y haciéndola más accesible. Como ejemplos podemos mencionar:
The Game Kitchen, desarrolladora española con sede en Sevilla que ha desarrollado Blasphemous. O juegos mundialmente conocidos como Undertale de Toby Fox, Stardew Valley de Eric "ConcernedApe" Barone. 



\section{Estado del arte}

El mundo del videojuego hoy en día es un ecosistema vivo y en constante evolución. La industria ha crecido enormemente, generando miles de millones de dólares en ingresos. Podemos destacar varias tendencias: 

- Diversificación de plataformas: Los juegos no solo se juegan en consolas y PCs, sino también en dispositivos móviles, ampliando el acceso y audiencia de los mismos.

- Desarrollo de juegos Indie's: Los desarrolladores independientes han ganado terreno, creando experiencias innovadoras que a menudo desafían las normas de la industria. Destacando "Hollow Knight" o "Celeste" entre otros, muy populares y aclamados tanto por su creatividad como jugabilidad.

- E-Sports: La competencia profesional en videojuegos se ha profesionalizado, con grandes torneos y ligas que atraen a millones de expectadores y producen grandes beneficios para las empresas detrás de los mismos. Destacan juegos como "League of Legends", "Valorant" o "Fortnite" y "Clash Royale".

- Realidad Virtual y Aumentada: La tecnología VR y AR está en auge, ofreciendo experiencias inmersivas. Juegos como "Beat Saber" o "Pokemon GO" muestran el potencial de estas tecnologías.

- Modelos de Negocio: Si bien se pueden asemejar al modelo de cualquier producto de entretenimiento más convencional como libros, de un solo pago, el videojuego permite nuevos modelos como las microtransacciones a cambio de beneficios o cosméticos dentro del juego o las subscripciones como los servicios de Netflix u otras plataformas de subscripción de pago, donde el contenido se va actualizando periódicamente. Vease las plataformas como Xbox Game Pass o PlayStation Plus. Aparte de los juegos que se actualizan en contenido como "World of Warcraft" y sus expansiones donde añaden mundos nuevos y misiones a completar. 
Grandes empresas dedican parte de sus esfuerzos dentro de este ámbito de los videojuegos. Colosos como Sony, Microsoft, Steam, Epic Games o incluso Amazon agrupan a gran cantidad de estudios que desarrollan y publican sus juegos en sus plataformas.

- Narrativa y Experiencia: Los videojuegos se están reconociendo cada vez más como una forma de arte, con narrativas profundas, experiencias emocionales y gráficos realistas como pueden ser "The Last Of Us" o "Baldur's Gate 3" con sus detallados mundos postapocalíptico y fantástico respectivamente.

- Inclusividad y diversidad: Hay un esfuerzo creciente por hacer que los juegos sean más inclusivos, con personajes diversos y narrativas que abordan temas sociales relevantes hoy en día.

- Educación: Algunas desarrolladoras han visto el potencial educativo de los videojuegos no solo para niños sino también para el desarrollo, formación y práctica de otros sectores como la salud. Encontramos ejemplos como "Minecraft Education" o "VR Surgery Simulator".

Esta imagen proporciona una perspectiva muy buena para el mundo del videojuego en los próximos años frente a la oleada de despidos y estudios cerrados que ha habido durante los últimos años .\href{https://publish.obsidian.md/vg-layoffs/Archive/2024} y \href{https://www.gamesindustry.biz/topics/studio-closures}. Este efecto se le atribuye a la estabilización del mercado tras la inflación que sufrió a partir de la pandemia del 2020 donde se popularizó masivamente el uso de los videojuegos como método de entretenimiento haciendo que se crearan estudios masivamente y se produjeran videojuegos por encima de las espectativas.

Sin embargo, el mundo del videojuego está lejos de desaparecer, cada año obteniendo más beneficios y generando productos de cada vez mayor calidad como el aclamado por la crítica "Baldur's Gate 3" que ha superado todas las espectativas tanto en ventas como en calidad a la hora de producción. Incluso poniendo en jaque a muchos estudios que publican sus juegos en estados que podrían ser casi calificados como "injugables".

\section{Premisa del proyecto}
El proyecto se basa en la premisa de un desarrollo de videojuego completo hecho en Unity. Para lo cual se ha propuesto un videojuego de cartas por turnos contra la máquina. Estas características han sido escogidas bajo previa aprobación del tutor del proyecto, coincidiendo con los criterios proporcionados para la correcta realización del trabajo.

Se toma el desarrollo de videojuego como reto personal y profesional del autor. Es un ámbito complejo que requiere de amplios conocimientos y habilidades tanto técnicas, creativas como sociales o humanas. Adquiriendo experiencia en áreas como la programación o la gestión de proyectos. Además, como se afronta desde la perspectiva de un novato dentro del ámbito del videojuego, se aprenderá desde cero.

Se espera un desarrollo de habilidades transferibles como la resolución de problemas y gestión de tiempo, altamente valoradas en la industria tecnológica entre otras. Así como el inicio de contrucción de un portafolio de proyectos personales que puedan ayudar en el futuro a respaldar mi experiencia y habilidades.

Por último destacar la satisfacción personal de realizar un proyecto de esta índole, siendo los videojuegos una pasión para mí y viendo realizado en forma de experiencia jugable una idea propia.