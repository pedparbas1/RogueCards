%!TEX root =  tfg.tex
\chapter{Contexto}

\begin{quotation}[Novelist]{Ernest Hemingway (1899--1961)}
The good parts of a book may be only something a writer is lucky enough to overhear or it may be the wreck of his whole damn life -- and one is as good as the other.
\end{quotation}

\begin{abstract}
Resumen de lo que va a ocurrir en el capítulo. ¿Cuál es el objetivo que tenemos con este capítulo?
\end{abstract}

\section{El mundo del Videojuego}

 El \cite[Videojuego]{wiki:Videojuego} es un ámbito relativamente joven si se compara con otros tipos de entretenimiento o (arte como es considerado por algunos) que han existido a lo largo de la historia contando con apenas 75 años desde la creación de las primeras máquinas que ofrecían una experiencia interactiva que apenas podían considerarse como videojuegos.


Hay que ir poco a poco acotando el contexto donde se desarrolla el proyecto. No se debe sobreentender que el evaluador de la memoria sabe del tema. Escribid el texto para la abuela.

\section{Subcontexto}

\section{Subsubcontexto}

\section{Estado del arte}

Cómo se encuentra la industria hoy en día a nivel económico y tecnológico.