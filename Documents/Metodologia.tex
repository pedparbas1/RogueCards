%!TEX root =  tfg.tex
\chapter{Metodología}

\begin{quotation}[Novelist]{Ernest Hemingway (1899--1961)}
The good parts of a book may be only something a writer is lucky enough to overhear or it may be the wreck of his whole damn life -- and one is as good as the other.
\end{quotation}

\begin{abstract}
Resumen de lo que va a ocurrir en el capítulo. ¿Cuál es el objetivo que tenemos con este capítulo?
\end{abstract}

\section{Estructura organizacional del proyecto}
En este caso se presenta un proyecto fin de grado individual de forma que el desarrollo no conlleva ninguna separación de responsabilidades. El estudiante llevará a cabo las labores tanto de desarrollador como de analista y gestor de proyecto. Produciendo así tanto el código a desarrollar como los informes y reportes generados por el desarrollo así como la contabilización y gestión de recursos y horas destinadas al proyecto.
¿Se hace en grupo? En caso afirmativo, ¿cuál va a ser la responsabilidad de cada uno?

\section{Metodología de desarrollo}
La metodología a usar será "Feature-Driven Development". Esta consiste en un proceso de desarrollo de software iterativa e incremental incluída en lo que se conoce en la industria como métodologías "Agile".
Estas metodologías se centran en rasgos generales en aportar valor a los "clientes" (en este caso los potenciales compradores de lo que sería el videojuego terminado) primándolo por encima de la generación de documentación, lo cual estaría más enfocado al mantenimiento y mejora de los sistemas que conforman el juego.

\section{Descripción detallada}
\begin{enumerate}
		\item Desarrollo del modelo general: 
			El proyecto comienza con una demostración a alto nivel del contexto y alcance del proyecto. Luego se dividen en sistemasmás pequeños para los cuales se generan modelos más detallados. Estos modelos más pequeños se estudian y según su viabilidad se implementan y mezclan con el resto del modelo general.
		\item Lista de funcionalidades:
			El conocimiento recopilado durante esta fase inicial se usa para identificar unalista de características que descomponenfuncionalmente cada una de las areas descritas anteriormente y las divide en piezas que pueden considerarse como unidades de desarrollo.
		\item Planear por funcionalidad:
			Tras completar la lista, el siguiente paso es producir el plan de desarrollo y asignar las funcionalidades a desarrollar como clases a los programadores.
		\item Diseñar por funcionalidad:
			Un diseño es producido para cada funcionalidad. No debe llevar más de dos semanas desarrollar uno de estos conjuntos de funcionalidades. Para su desarrollo se generan diagramas de secuencia detallados por cada una de las funcionalidades.
		\item Construir una funcionalidad:
			Tras una inspección del diseño satisfactoria por cada actividad para producir una funcionalidad, el desarrollador reliza su trabajo. Tras las pruebas unitarias y las inspecciones de código, la funcionalidad desarrollada se mezcla en la rama principal.
\end{enumerate}




Indicar en qué metodología nos basamos, explicarla brevemente y luego adaptarla a nuestras necesidades. Cada una de estas cuestiones debe ser una subsección.
