%!TEX root =  tfg.tex
\chapter{Planificación}

\begin{quotation}[Novelist]{Ernest Hemingway (1899--1961)}
The good parts of a book may be only something a writer is lucky enough to overhear or it may be the wreck of his whole damn life -- and one is as good as the other.
\end{quotation}

\begin{abstract}
Resumen de lo que va a ocurrir en el capítulo. ¿Cuál es el objetivo que tenemos con este capítulo?
\end{abstract}

\section{Resumen temporal del proyecto}

\begin{table*}[htb]
	\centering
	\begin{coolTable}{ll}{2}
{Resumen del proyecto}
	\textbf{Fecha de inicio}&06/07/2024\\
	\textbf{Fecha de fin}&09/10/2024\\
	\textbf{Periodicidad de las revisiones}1 previa a la entrega\\
	\textbf{Carga de trabajo semanal}&24,5 horas\\
	\textbf{Horas totales previstas}&300 horas\\ % entre 25-30 horas por crédito
	\textbf{Horas finales}&300 horas\\
	\end{coolTable}
	\caption{Tabla resumen de tiempos y planificación}
\end{table*}

\section{Planificación inicial}

Aquí un desglose de las iteraciones, comienzo y fin de cada una:

\begin{table*}[htb]
	\centering
	\begin{coolTable}{ll}{2}
{Resumen de iteraciones}
	\textbf{Iteración 1}&06/07/24 a 21/07/24\\
	\textbf{Iteración 2}&22/07/24 a 07/08/24\\
	\textbf{Iteración 3}&08/08/24 a 26/08/24\\ 
	\textbf{Iteración 4}&27/08/24 a 10/09/24\\
	\textbf{Iteración 5}&11/09/24 a 01/10/24\\
	\end{coolTable}
	\caption{Planificación temporal de iteraciones}
\end{table*}

El criterio seguido para la separación de los periodos de iteración es el siguiente:
\begin{itemize}
	\item [Vacaciones:] Debido a que trabajo a la vez que desarrollo el proyecto, se va a tomar como referencia las vacaciones aprobadas y definir toda la planificación en base a esas dos semanas que se ven representadas en la tercera iteración. Estas representarán la mayor carga de trabajo debido al aumento de tiempo disponible para dedicarle al proyecto.
	\item[Empleo:] En el caso de coincidir con una semana de aumento de la carga de trabajo, como medida preventiva, se ha decidido alargar las iteraciones hasta un máximo de 3 días naturales 
	\item [Semanas naturales:] Se intentará terminar cada iteración con cada semana natural, puesto que los días no laborables se podrá dedicar más tiempo a resolver problemas, probar e integrar los progresos que hayan habido durante la iteración.
\end{itemize}

\subsection{Desglose de iteraciones según planificación inicial}

\subsubsection{Iteración 1}
\textbf{ESTE CAPÍTULO DEBE ESCRIBIRSE AL COMIENZO DEL PROYECTO}

\section{Informe de tiempos del proyecto}

Lo mismo que el anterior pero con datos reales. Ver Tabla \ref{tab:InformeTiempos}.

\begin{table*}[htb]
	\centering
	\begin{coolTable}{ll}{2}
{Resumen de iteraciones}
	\textbf{Iteración 1}&10/10/14 a 21/10/14\\
	\textbf{Iteración 2}&21/10/14 a 15/11/14\\
	\textbf{...}&dd/mm/aa a dd/mm/aa\\
	\end{coolTable}
	\caption{Planificación temporal de iteraciones\label{tab:InformeTiempos}}
\end{table*}

Justificar los retrasos de forma detallada aquí para cada una de las iteraciones. Explicar las razones.

problemas durante el drag de las cartas ' displacement
implementacion arquitectura de soa